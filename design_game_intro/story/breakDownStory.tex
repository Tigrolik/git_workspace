\documentclass{article}

\textheight = 600pt

\pagestyle{empty}

\begin{document}

\section*{Breaking Down: Movie ``Inception''}

\subsubsection*{Spine}

A great expert in subconscious infiltration is hired to break the energy
monopoly conglomerate and assembles his team for the task. They exploit the idea
of dream within a dream to go down several layers of sleeping in order to seed
the idea of dissolving a company into the brain of the company owner's son.

\subsubsection*{Act I}

Dom and Arthur (``extractors'') use subconsious to retrieve information while
experiencing a shared dream with their target (Saito). That demonstrates how
good they are as well as introduces an obstacle in the form of Dom's wife (Mal).

Saito later approaches Dom to hire him for a very subtle and difficult task in
exchange for a favour --- Saito will help Dom to see his children who live
separately from their father due to unfortunate circumstances.

\subsubsection*{Act II}

Dom assembles his team of best experts to perform this extremely sophisticated
task. They have got the only one chance to accomplish their job on a 10--hour
flight. The team submerges in the dream together with their target --- son of
the monopoly energy company owner. Waking up from the dreams should be done with
very neat precision. As the characters plundge deeper in the dreams, the tension
rises and in the what was supposed to be the last layer Dom's wife appears and
intervenes by nearly killing the target. With the last effort and defibrillator
they succeed to bring the target to life in one more layer down. Thus they
succeed to plant the idea in the head of the owner's son and perfectly time
their waking. Dom convinces his wife to stop pursuing him and that she is only
his imagination. However, the waking of Dom remains uncertain as he falls even
deeper.

\subsubsection*{Act III}

The mission is accomplished and the energy monopoly will be stopped for now. We
see with Dom's eyes that he wakes up and sees everyone is happy and satisfied as
well as Saito's promise is fulfilled --- he meets his children and is happy.
However, the fact whether he has truly woken up or he sees this in the eternal
dream remain unresolved.

\subsubsection*{Conclusion}

This story is interesting to me (as, I guess, to many other people) because it
shows how our brains may actually work. It might be another story about mind
games, however, its demonstration is alluring. Another side is the effect that
people may believe that their subconscious is penetrable and be intimidated.
The story is also about measuring things and timing precisely which keeps us
watching to see if the team succeeds to make it in time (even though, deep
inside we know they will succeed).

\end{document}
